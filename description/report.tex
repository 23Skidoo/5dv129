\documentclass[10pt, oneside]{article}

\usepackage{amsmath}
\usepackage{amssymb}
\usepackage[utf8]{inputenc}
\usepackage[swedish]{babel}
\usepackage{enumerate}
\usepackage{titling}
\usepackage[nottoc, notlof]{tocbibind}
\usepackage[pdftex]{graphicx}
\usepackage[kerning,spacing]{microtype}
\usepackage{verbatim}
\usepackage{color}
\usepackage{listings}

\usepackage[bookmarksnumbered, unicode, pdftex]{hyperref}

\author{Mikhail Glushenkov, \texttt{<c05mgv@cs.umu.se>}}

\title{A Cross-Platform Scalable I/O Manager for GHC}

\newcommand{\unit}[1]{\ensuremath{\, \mathrm{#1}}}

\begin{document}
\thispagestyle{empty}

\section*{A Cross-Platform Scalable I/O Manager for GHC}

GHC is an industrial-strength compiler for Haskell, a popular lazy statically
typed functional language. One of the areas GHC's implementation of Haskell
shines at is concurrency: the programmer can create a very large number of cheap
lightweight threads that are internally executed using a smaller number of
operating system (OS) threads. While this model's simplicity is attractive from
the programmer's perspective, it complicates the runtime system (RTS)
implementation considerably.

One part of the RTS that has seen active development activity recently is the
I/O manager. When a Haskell thread does a blocking I/O operation, like writing
to a socket or reading from a file, it should not be allowed to block the OS
thread it's executing on. The job of the I/O manager is to handle all I/O
activity behind the scenes and notify blocked Haskell threads about completed
I/O operations.

GHC version 7.0 for the first time included a new I/O manager~\cite{bib:tibell}
that improved performance and allowed I/O to scale to a much larger number of
threads than previously. In version 7.8, the implementation has been optimised
even further~\cite{bib:mio} and now scales effectively up to 48 cores. Sadly,
these improvements are available only to users on Unix-like platforms due to the
differences between I/O event models on Windows (I/O completion ports) and Unix
(\texttt{epoll}/\texttt{kqueue}/\texttt{poll}).

This proposal is essentially about fixing the GHC issue \#7353~\cite{bib:ticket}
and making the performance and scalability improvements enjoyed by users of GHC
on Unix available to Haskell programmers on Windows. Due to time constraints,
the project will initially focus only on socket I/O, ignoring file I/O for the
time being. The project will involve redesigning the I/O manager API to support
both IOCP and \texttt{epoll}-like event models; adding support for IOCP to the
I/O manager code; adding support for new Windows I/O manager features to the
standard networking library; and finally, optimising the resulting
implementation. There is already some unfinished work in this area that this
project can build on (see discussion in~\cite{bib:ticket}). The goal of the
project is to produce a patchset suitable for inclusion in GHC (maybe after some
further polishing).

\renewcommand\refname{References}

\begin{thebibliography}{9}

\bibitem[Tibell 2010]{bib:tibell}
  \href{http://research.google.com/pubs/pub36841.html}{\emph{Scalable I/O Event Handling for GHC}}\\
  \newblock Bryan O'Sullivan \& Johan Tibell\\
  \newblock Proceedings of the 2010 ACM SIGPLAN Haskell Symposium (Haskell'10).

\bibitem[Voellmy 2013]{bib:mio}
  \href{http://haskell.cs.yale.edu/wp-content/uploads/2013/08/hask035-voellmy.pdf}{\emph{Mio:
      A High-Performance Multicore IO
      Manager for GHC}}\\
  \newblock Andreas Voellmy et al.\\
  \newblock Proceedings of the 2013 ACM SIGPLAN symposium on Haskell (Haskell'13).

\bibitem[GHC Trac]{bib:ticket} \emph{GHC Trac issue \#7353}\\
  \newblock \url{https://ghc.haskell.org/trac/ghc/ticket/7353}

\end{thebibliography}

\end{document}
