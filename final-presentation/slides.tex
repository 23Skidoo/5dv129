\documentclass{beamer}

\mode<presentation>
{
  \usetheme{Warsaw}
  %\usecolortheme{seahorse}
  % or ...

  \setbeamercovered{transparent}
  % or whatever (possibly just delete it)
}


\usepackage[english]{babel}
% or whatever

\usepackage[latin1]{inputenc}
% or whatever

\usepackage{amssymb}
\usepackage{times}
%\usepackage[T1]{fontenc}
% Or whatever. Note that the encoding and the font should match. If T1
% does not look nice, try deleting the line with the fontenc.


\title{A Cross-Platform Scalable I/O Manager for GHC}

\author{Mikhail~Glushenkov}

\date{\today}

\subject{A Cross-Platform Scalable I/O Manager for GHC}

\begin{document}

\begin{frame}
  \titlepage
\end{frame}

\begin{frame}{Summary}
  \emph{Main idea:} Make it possible to program network server applications in
  Haskell that work on Windows and support a large number of concurrent
  connections.
\end{frame}

\begin{frame}{Outline}
  \tableofcontents
  % You might wish to add the option [pausesections]
\end{frame}

\section {Background}

\begin{frame}{Haskell}
  \begin{itemize}
  \item A popular functional language.
  \item Powerful static type system.
  \item Focus on immutability.
  \item Lazy (call-by-need).
  \end{itemize}
\end{frame}

\begin{frame}{GHC}
  \begin{itemize}
  \item The Glasgow Haskell Compiler.
  \item An industrial-strength implementation of Haskell.
  \item A lot of extensions.
  \item Compiles to fast native code.
  \item \emph{Open source}.
  \item \emph{Great support for concurrency}.
  \end{itemize}
\end{frame}

\begin{frame}{Concurrent Haskell}
  \begin{itemize}
  \item ``Green threads''.
  \item Multiplexed on a small number of OS threads.
  \item Lightweight: low memory/task switching cost.
  \item Easy to create a large number of threads.
  \item ``Large'' as in 100K-1M.
  \item Supports multicore.
  \end{itemize}
\end{frame}

\begin{frame}[fragile]{Code Example}
\begin{verbatim}
    main = do
        listenSock <- startListenSock
        forever $ do
            sock <- accept listenSock
            forkIO $ worker sock

    worker sock = do
       -- Do some I/O ...
       send sock "question"
       answer <- recv sock
\end{verbatim}
\end{frame}

\begin{frame}{The Problem of I/O}
  \begin{itemize}
  \item An I/O call should only block the Haskell thread.
  \item \textbf{Not} the underlying OS thread.
  \item I/O should also scale well.
  \end{itemize}
\end{frame}

\begin{frame}{Solution: I/O Manager}
  \begin{itemize}
  \item Part of the runtime support system.
  \item A dedicated thread for handling I/O.
  \item Uses a high-performance event-based I/O API (\texttt{epoll}/\texttt{kqueue}).
  \item Basically, an event processing loop.
  \item Heavily optimised, see the Mio paper.
  \item Scales to a large number of cores.
  \end{itemize}
\end{frame}

\begin{frame}{Problem: Windows support}
  \begin{itemize}
  \item No I/O manager on Windows.
  \item I/O uses safe FFI calls.
  \end{itemize}
\end{frame}

\begin{frame}{Safe foreign calls}
  \begin{itemize}
  \item A foreign call that runs in its own OS thread.
  \item Does not block other Haskell threads.
  \item More overhead.
  \item Separate OS thread for each concurrent I/O call.
  \end{itemize}
\end{frame}

\begin{frame}[fragile]{Code Example}
  Non-interruptible:
\begin{verbatim}
  timeout 5000000 $ connectTo "google.com"
                    (PortNumber 1234)
\end{verbatim}
  \begin{itemize}
  \item On Linux: timeout exception.
  \item On Windows: hangs.
  \end{itemize}
\end{frame}

\section {New I/O Manager Design}

\begin{frame}{\texttt{epoll} vs. IOCP}
  \begin{itemize}
  \item \texttt{epoll}: Linux API for scalable I/O.
  \item Wait for all descriptors to become ready and then start doing I/O.
  \item IOCP: Windows API for scalable I/O.
  \item Inititate all I/O requests asynchronously and wait for completion.
  \end{itemize}
\end{frame}

\begin{frame}{Haskell interface to Unix I/O manager}
\begin{itemize}
    \item \texttt{threadWaitRead  :: Fd -> IO ()}
    \item \texttt{threadWaitWrite :: Fd -> IO ()}
\end{itemize}
Blocks until the file descriptor becomes ready.

Used in the standard library to implement \texttt{send}/\texttt{recv} et al.
\end{frame}

\begin{frame}[fragile]{Haskell interface to Windows I/O manager}
\begin{verbatim}
withOverlapped  :: HANDLE -> StartCallback
                -> CompletionCallback -> IO ()

type StartCallback  :: LPOVERLAPPED -> IO ()

type CompletionCallback  :: ErrCode -> DWORD -> IO ()
\end{verbatim}
  Initiates the I/O with \texttt{StartCallback}, blocks until the completion
  arrives, calls \texttt{CompletionCallback}.

  Problem: the standard library must be rewritten.
\end{frame}

\begin{frame}{Prior Art}
\begin{itemize}
    \item Existing attempt at Windows I/O manager:
    \item \url{https://ghc.haskell.org/trac/ghc/ticket/7353}
    \item \url{https://github.com/joeyadams/haskell-iocp}
    \item Incomplete and slower than the status quo.
\end{itemize}
\end{frame}

\begin{frame}{New Design}
\begin{itemize}
    \item Builds upon the previous attempt.
    \item Adds a number of optimisations.
    \item Removes support for Windows XP.
    \item Ideas from the Mio paper.
\end{itemize}
\end{frame}

\begin{frame}{Bound thread pool}
  \begin{itemize}
    \item Made everything slower.
    \item Only needed to support Windows XP.
    \item Solution: remove.
    \item Use \texttt{CancelIOEx}.
    \item Use reference-counting for keeping OS threads with pending I/O alive.
\end{itemize}
\end{frame}

\begin{frame}{GetQueuedCompletionStatusEx}
  \begin{itemize}
    \item Removes multiple completions with a single system call.
    \item Less overhead.
    \item Again, Windows Vista+ only.
\end{itemize}
\end{frame}

\begin{frame}{Scalable registrations}
\begin{itemize}
  \item Old design used stable pointers.
  \item Problem: global lock.
  \item New design uses a two-level hash table.
  \item Locking only needed for the second level.
  \item Idea from the Mio paper.
\end{itemize}
\end{frame}

\begin{frame}{Separate timeout manager thread}
  \begin{itemize}
    \item (WIP)
    \item Makes the I/O manager thread more responsive.
    \item Can avoid safe foreign calls to \texttt{GetQueuedCompletionStatusEx}.
    \item Idea from the Mio paper.
\end{itemize}
\end{frame}

\begin{frame}{Per-core I/O manager threads}
  \begin{itemize}
    \item (WIP)
    \item Lets us scale to a large number of cores.
    \item Idea from the Mio paper.
\end{itemize}
\end{frame}

\section {Evaluation (Benchmarks)}

\begin{frame}{PongServer: Throughput}
  10K requests, requests per second:
\begin{tabular}{ | l | l | l | }
  \hline
  Connections & \texttt{pong} & \texttt{pong-baseline} \\
  \hline
  10 & 10700 & 10500 \\
  \hline
  100 & 10700 & 10200 \\
  \hline
  1000 & 8400 & 8500 \\
  \hline
  10000 & 5300 & 5200 \\
  \hline
\end{tabular}
\end{frame}

\begin{frame}{PongServer: Latency}
  10K requests, processing time per request (ms):
\begin{tabular}{ | l | l | l | }
  \hline
  Connections & \texttt{pong} & \texttt{pong-baseline} \\
  \hline
  10 &  0.93 &  0.94 \\
  \hline
  100 &  9.0 &  9.8 \\
  \hline
  1000 & 120.1 & 115.7 \\
  \hline
  10000 & 1800 & 1900 \\
  \hline
\end{tabular}
\end{frame}

\begin{frame}{FileServer: Throughput}
  10K requests, requests per second:
\begin{tabular}{ | l | l | l | }
  \hline
  Connections & \texttt{file} & \texttt{file-baseline} \\
  \hline
  10 & 5600 & 9200 \\
  \hline
  100 & 5700 & 8800 \\
  \hline
  1000 & 1250 & 5500 \\
  \hline
  10000 & 560 & 1400 \\
  \hline
\end{tabular}
\end{frame}

\section {Conclusions \& Future Work}

\begin{frame}{Current Status}
  \begin{itemize}
  \item Only a proof of concept.
  \item Competitive performance (with socket I/O).
  \item Improved correctness.
  \item Only supports socket I/O (no file I/O).
  \item Not fully integrated with GHC.
  \end{itemize}
\end{frame}

\begin{frame}{Future Work}
  \begin{itemize}
  \item A separate timeout manager thread.
  \item Per-core I/O manager threads.
  \item More benchmarks.
  \item Support for file I/O.
  \item Propose for inclusion in GHC.
  \end{itemize}
\end{frame}

\section {Further Reading}

\begin{frame}{Further Reading: }
  \begin{itemize}
  \item Thesis draft and these slides are available on GitHub:
  \item \url{https://github.com/23Skidoo/5dv129}
  \item As well as the implementation:
  \item \url{https://github.com/23Skidoo/haskell-iocp}
  \item \url{https://github.com/23Skidoo/ghc} (the
    \texttt{ghc-8.0-max-spare-workers} branch)
  \end{itemize}
\end{frame}

\begin{frame}{Further Reading: GHC I/O Manager}
  \begin{itemize}
  \item \href{http://research.google.com/pubs/pub36841.html}{\emph{Scalable I/O Event Handling for GHC}}\\
    Bryan O'Sullivan \& Johan Tibell\\
  \item \href{http://haskell.cs.yale.edu/wp-content/uploads/2013/08/hask035-voellmy.pdf}{\emph{Mio:
        A High-Performance Multicore IO
        Manager for GHC}}\\
    Andreas Voellmy et al.\\
  \item \emph{GHC Trac issue \#7353}\\
    \url{https://ghc.haskell.org/trac/ghc/ticket/7353}
  \end{itemize}
\end{frame}

\begin{frame}{Further Reading: Background}
  On the programming model:
  \begin{itemize}
  \item
    \emph{Parallel and Concurrent Programming in Haskell} (O'Reilly book)\\
    by Simon Marlow
  \end{itemize}
  On the runtime system:
  \begin{itemize}
  \item
    \href{http://community.haskell.org/~simonmar/papers/multicore-ghc.pdf}{\emph{Runtime
    Support for Multicore Haskell}}\\
    Simon Marlow, Simon Peyton Jones, Satnam Singh.
  \item
    \href{http://community.haskell.org/~simonmar/papers/conc-ffi.pdf}{\emph{Extending
    the Haskell Foreign Function Interface with Concurrency}}\\
    Simon Marlow, Simon Peyton Jones, Wolfgang Thaller
  \end{itemize}
\end{frame}

\end{document}
