\documentclass{beamer}

% \documentclass[handout]{beamer}
% ^ For generating handouts.

\mode<handout>
{
  \usepackage{pgf}
  \usepackage{pgfpages}

\pgfpagesdeclarelayout{4 on 1 boxed}
{
  \edef\pgfpageoptionheight{\the\paperheight}
  \edef\pgfpageoptionwidth{\the\paperwidth}
  \edef\pgfpageoptionborder{0pt}
}
{
  \pgfpagesphysicalpageoptions
  {%
    logical pages=4,%
    physical height=\pgfpageoptionheight,%
    physical width=\pgfpageoptionwidth%
  }
  \pgfpageslogicalpageoptions{1}
  {%
    border code=\pgfsetlinewidth{2pt}\pgfstroke,%
    border shrink=\pgfpageoptionborder,%
    resized width=.5\pgfphysicalwidth,%
    resized height=.5\pgfphysicalheight,%
    center=\pgfpoint{.25\pgfphysicalwidth}{.75\pgfphysicalheight}%
  }%
  \pgfpageslogicalpageoptions{2}
  {%
    border code=\pgfsetlinewidth{2pt}\pgfstroke,%
    border shrink=\pgfpageoptionborder,%
    resized width=.5\pgfphysicalwidth,%
    resized height=.5\pgfphysicalheight,%
    center=\pgfpoint{.75\pgfphysicalwidth}{.75\pgfphysicalheight}%
  }%
  \pgfpageslogicalpageoptions{3}
  {%
    border code=\pgfsetlinewidth{2pt}\pgfstroke,%
    border shrink=\pgfpageoptionborder,%
    resized width=.5\pgfphysicalwidth,%
    resized height=.5\pgfphysicalheight,%
    center=\pgfpoint{.25\pgfphysicalwidth}{.25\pgfphysicalheight}%
  }%
  \pgfpageslogicalpageoptions{4}
  {%
    border code=\pgfsetlinewidth{2pt}\pgfstroke,%
    border shrink=\pgfpageoptionborder,%
    resized width=.5\pgfphysicalwidth,%
    resized height=.5\pgfphysicalheight,%
    center=\pgfpoint{.75\pgfphysicalwidth}{.25\pgfphysicalheight}%
  }%
}


  \pgfpagesuselayout{4 on 1 boxed}[a4paper, border shrink=5mm, landscape]
  \nofiles
}

\mode<presentation>
{
  \usetheme{Warsaw}
  %\usecolortheme{seahorse}
  % or ...

  \setbeamercovered{transparent}
  % or whatever (possibly just delete it)
}


\usepackage[english]{babel}
\usepackage{tikz}
\usepackage{amssymb}
\usepackage{times}
\usepackage{soul}
\usepackage{hyperref}
%\usepackage[T1]{fontenc}
% Or whatever. Note that the encoding and the font should match. If T1
% does not look nice, try deleting the line with the fontenc.

\def\checkmark{\tikz\fill[scale=0.4](0,.35) -- (.25,0) -- (1,.7) -- (.25,.15) -- cycle;}

\title{A Cross-Platform Scalable I/O Manager for GHC}

\author{Mikhail~Glushenkov}

\date{\today}

\subject{A Cross-Platform Scalable I/O Manager for GHC}

\begin{document}

\begin{frame}
  \titlepage
\end{frame}

\begin{frame}{Outline}
  \tableofcontents
  % You might wish to add the option [pausesections]
\end{frame}

\section {Refresher}

\begin{frame}{Main idea}
  \emph{Main idea:} Make it possible to program network server applications in
  Haskell that work on Windows and support a large number of concurrent
  connections.
\end{frame}

\begin{frame}{Concurrent Haskell}
  \begin{itemize}
  \item ``Green threads''.
  \item Multiplexed on a small number of OS threads.
  \item Lightweight: low memory/task switching cost.
  \item Easy to create a large (100K-1M) number of threads.
  \item Supports multicore.
  \end{itemize}
\end{frame}

\begin{frame}{The Problem of I/O}
  \begin{itemize}
  \item I/O calls should only block Haskell threads.
  \item I/O should also scale well.
  \item Haskell I/O APIs are synchronous.
  \item (Blocking \texttt{send}/\texttt{recv} calls).
  \item High-performance I/O APIs use event-based style.
  \item (\texttt{epoll}, \texttt{kqueue}).
  \end{itemize}
\end{frame}

\begin{frame}{Solution: I/O Manager}
  \begin{itemize}
  \item Part of the runtime support system.
  \item A dedicated \textbf{Haskell} thread for handling I/O.
  \item Runs an event loop implemented on top of a non-blocking I/O API.
  \item Processes all I/O initiated by Haskell threads.
  \item No Windows support.
  \end{itemize}
\end{frame}

\section{Current Status}

\begin{frame}[fragile]{Main Result}
  Network I/O is now interruptible:
\begin{verbatim}
  timeout 5000000 $ connectTo "google.com"
                    (PortNumber 1234)
\end{verbatim}
  \begin{itemize}
  \item On Linux: exception.
  \item On Windows: \st{hangs forever} exception.
  \item \textit{(provided that a non-standard network library is used)}
  \end{itemize}
\end{frame}

\begin{frame}{Time Plan (Original)}
  \begin{itemize}
  \item \st{Phase 1: Make it work.} \checkmark
  \item Phase 2: Make it fast.
  \end{itemize}
\end{frame}

\begin{frame}{Phase 1 Report}
  \begin{itemize}
  \item Studied GHC source code and IOCP documentation.
  \item Studied the code of the original (unfinished) implementation.
  \item Ported the original implementation to GHC 8.0.
  \item Code available on \url{https://github.com/23Skidoo/ghc}
    (\texttt{ghc-8.0-windows-iocp} branch).
  \item Haven't yet started writing the report.
  \end{itemize}
\end{frame}

\section {What's next}

\begin{frame}{How to make it fast?}
  \begin{itemize}
    \item Drop support for versions of Windows before Vista.
    \item (Lets us simplify the code and use new API features).
  \end{itemize}
\end{frame}

\begin{frame}{How to make it fast?}
  \begin{itemize}
    \item Main problem: context switching costs.
    \item I/O manager currently runs in its own OS thread.
    \item Try to get rid of the bound thread restriction.
    \item Use refcounting to prevent OS threads that initiated I/O from exiting.
    \item Failing that, try to integrate with the scheduler.
    \item Schedule threads so as to minimise context-switching costs.
  \end{itemize}
\end{frame}

\begin{frame}{Time Plan (Updated)}
  Second milestone: make it fast
  \begin{itemize}
    \item Meaning of ``fast'': no worse than status quo on Windows.
    \item Deadline: second seminar.
    \item Week 4: First report chapter, benchmarks, drop XP support.
    \item Week 5-7: I/O manager optimisation, work on the report.
    \item Week 7: Draft report.
    \item Remaining time will be spent on polishing.
  \end{itemize}
\end{frame}

\begin{frame}{Risk Management}
  Risk management strategy: make it as minimal as possible
  \begin{itemize}
    \item Focus only on Windows (\textbf{Vista+}) support.
    \item Focus only on socket I/O.
    \item Use a lightweight networking library for testing \& benchmarking.
    \item (instead of patching the standard one).
  \end{itemize}

\end{frame}

\end{document}
