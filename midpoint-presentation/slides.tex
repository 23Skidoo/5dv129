\documentclass{beamer}

\mode<presentation>
{
  \usetheme{Warsaw}
  %\usecolortheme{seahorse}
  % or ...

  \setbeamercovered{transparent}
  % or whatever (possibly just delete it)
}


\usepackage[english]{babel}
% or whatever

\usepackage[latin1]{inputenc}
% or whatever

\usepackage{amssymb}
\usepackage{times}
%\usepackage[T1]{fontenc}
% Or whatever. Note that the encoding and the font should match. If T1
% does not look nice, try deleting the line with the fontenc.


\title{A Cross-Platform Scalable I/O Manager for GHC}

\author{Mikhail~Glushenkov}

\date{\today}

\subject{A Cross-Platform Scalable I/O Manager for GHC}

\begin{document}

\begin{frame}
  \titlepage
\end{frame}

\begin{frame}{Summary}
  \emph{Main idea:} Make it possible to program network server applications in
  Haskell that work on Windows and support a large number of concurrent
  connections.
\end{frame}

\begin{frame}{Outline}
  \tableofcontents
  % You might wish to add the option [pausesections]
\end{frame}

\section {Background}

% Bakgrund
% –
% Kontext (Avvikelser, olöst, dåliga algoritmer, ny approach...
%  )
% –
% Vad behöver jag läsa in mig på?
% –
% Vilken litteratur ska läsas?

\begin{frame}{Haskell}
  \begin{itemize}
  \item A popular functional language.
  \item Powerful static type system.
  \item Focus on immutability.
  \item Lazy (call-by-need).
  \end{itemize}
\end{frame}

\begin{frame}{GHC}
  \begin{itemize}
  \item The Glasgow Haskell Compiler.
  \item An industrial-strength implementation of Haskell.
  \item (``Haskell'' == ``GHC Haskell'' for practical purposes).
  \item A lot of extensions.
  \item Compiles to fast native code.
  \item \emph{Open source}.
  \item \emph{Great support for concurrency}.
  \end{itemize}
\end{frame}

\begin{frame}{Concurrent Haskell}
  \begin{itemize}
  \item ``Green threads''.
  \item Multiplexed on a small number of OS threads.
  \item Lightweight: low memory/task switching cost.
  \item Easy to create a large number of threads.
  \item ``Large'' as in 100K-1M.
  \item Easier to program than event-based style (e.g. NodeJS).
  \item Supports multicore.
  \end{itemize}
\end{frame}

\begin{frame}[fragile]{Code Example}
\begin{verbatim}
    main = do
        listenSock <- startListenSock
        forever $ do
            sock <- accept listenSock
            forkIO $ worker sock

    worker sock = do
       -- Do some I/O ...
       send sock "question"
       answer <- recv sock
\end{verbatim}
\end{frame}

\begin{frame}{The Problem of I/O}
  \begin{itemize}
  \item I/O calls should only block Haskell threads.
  \item I/O should also scale well.
  \item High-performance I/O APIs use event-based style.
  \item Example: \texttt{epoll}, \texttt{kqueue}.
  \end{itemize}
\end{frame}

\begin{frame}{Solution: I/O Manager}
  \begin{itemize}
  \item Part of the runtime support system.
  \item A dedicated OS thread for handling I/O.
  \item Uses an event-based I/O API.
  \item Presents a synchronous API to the programmer.
  \item (Blocking \texttt{send}/\texttt{recv} calls).
  \item With recent improvements, scales to 40 cores (at least).
  \end{itemize}
\end{frame}


\section {Problem Description}

% ●
% Problembeskrivning (lägg fokus här)
% –
% Vad ska besvaras?

\begin{frame}{Problem: Windows support}
  \begin{itemize}
  \item GHC on Windows doesn't have an I/O manager.
  \item Socket I/O use safe FFI calls.
  \item ``Safe FFI'': does not block other Haskell threads.
  \item Problem: not as scalable as an event-based I/O API.
  \item Problem: not interruptible.
  \end{itemize}
\end{frame}

\begin{frame}[fragile]{Code Example}
\begin{verbatim}
  timeout 5000000 $ connectTo "google.com"
                    (PortNumber 1234)
\end{verbatim}
  \begin{itemize}
  \item On Linux: exception.
  \item On Windows: hangs forever.
  \end{itemize}
\end{frame}

\section {Proposed Solution}

% ●
% Metoder
% –
% Hur ska problemet lösas?

\begin{frame}{Proposed Solution}
  \begin{itemize}
  \item Implement an I/O manager for Windows.
  \item Use Windows native event-based I/O API.
  \item Redesign GHC's internal APIs to support the Windows model.
  \item Integrate with the RTS scheduler to make it fast.
  \end{itemize}
\end{frame}

\begin{frame}{Prior Art}
  \begin{itemize}
  \item There is already a unfinished implementation of an I/O manager for
    Windows.
  \item See discussion in \emph{GHC Trac issue \#7353}
    \url{https://ghc.haskell.org/trac/ghc/ticket/7353}.
  \item My work will build on that.
  \item Aim: patchset suitable for inclusion in GHC.
  \item Or at least a proof of concept.
  \end{itemize}
\end{frame}

\begin{frame}{Challenges}
  \begin{itemize}
  \item Different API on Windows.
  \item Different limitations.
  \item There was already one (partially) failed attempt.
  \end{itemize}
\end{frame}

\begin{frame}{IOCP vs. epoll-based}
  \begin{itemize}
  \item IOCP: start all I/O operations asynchronously and wait for their
    completion.
  \item epoll: wait until one of the files is ``ready'' and the do the I/O
    operation.
  \end{itemize}
\end{frame}

\section {Time Plan}

% ●
% Tidsplan (var finns riskerna?)
% –
% Hur lång tid förväntas varje moment ta


\begin{frame}{Time Plan (Preliminary)}
  \begin{itemize}
  \item Phase 1: Make it work.
  \item Phase 2: Make it fast.
  \end{itemize}
\end{frame}

\begin{frame}{Time Plan (Preliminary)}
  First milestone: make it work
  \begin{itemize}
    \item Deadline: first seminar
    \item Week 1: Study the existing implementation, describe it in the report.
    \item Week 2-3: Port old code to a new GHC, benchmark \& test.
    \item Meaning of ``make it work'': interruptible socket I/O.
    \item Not necessarily fast.
  \end{itemize}
\end{frame}

\begin{frame}{Time Plan (Preliminary)}
  Second milestone: make it fast
  \begin{itemize}
    \item Deadline: second seminar.
    \item Week 4-7.
    \item Meaning of ``fast'': no worse than status quo on Windows.
    \item Hopefully: on par with \texttt{epoll}-based version.
    \item Hopefully: apply ideas from the Mio paper.
    \item The remaining time will be spent on polishing.
  \end{itemize}
\end{frame}

\begin{frame}{Risk Management}
  Risk management strategy: make it as minimal as possible
  \begin{itemize}
    \item Focus only on Windows support.
    \item Focus only on socket I/O.
    \item Use a lightweight networking library for testing \& benchmarking.
    \item (instead of patching the standard one).
    \item Re-evaluate after the first milestone.
  \end{itemize}
\end{frame}


\section {Further Reading}

\begin{frame}{Further Reading: GHC I/O Manager}
  \begin{itemize}
  \item \href{http://research.google.com/pubs/pub36841.html}{\emph{Scalable I/O Event Handling for GHC}}\\
    Bryan O'Sullivan \& Johan Tibell\\
  \item \href{http://haskell.cs.yale.edu/wp-content/uploads/2013/08/hask035-voellmy.pdf}{\emph{Mio:
        A High-Performance Multicore IO
        Manager for GHC}}\\
    Andreas Voellmy et al.\\
  \item \emph{GHC Trac issue \#7353}\\
    \url{https://ghc.haskell.org/trac/ghc/ticket/7353}
  \end{itemize}
\end{frame}

\begin{frame}{Further Reading: Background}
  On the programming model:
  \begin{itemize}
  \item
    \emph{Parallel and Concurrent Programming in Haskell} (O'Reilly book)\\
    by Simon Marlow
  \end{itemize}
  On the runtime system:
  \begin{itemize}
  \item
    \href{http://community.haskell.org/~simonmar/papers/multicore-ghc.pdf}{\emph{Runtime
    Support for Multicore Haskell}}\\
    Simon Marlow, Simon Peyton Jones, Satnam Singh.
  \item
    \href{http://community.haskell.org/~simonmar/papers/conc-ffi.pdf}{\emph{Extending
    the Haskell Foreign Function Interface with Concurrency}}\\
    Simon Marlow, Simon Peyton Jones, Wolfgang Thaller
  \end{itemize}
\end{frame}

\end{document}
