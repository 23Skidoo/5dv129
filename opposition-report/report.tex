\documentclass[10pt, oneside]{article}

\usepackage{amsmath}
\usepackage{amssymb}
\usepackage[utf8]{inputenc}
\usepackage[english]{babel}
\usepackage{enumerate}
\usepackage{titling}
\usepackage[nottoc, notlof]{tocbibind}
\usepackage[pdftex]{graphicx}
\usepackage[kerning,spacing]{microtype}
\usepackage{verbatim}
\usepackage{color}
\usepackage{listings}

\usepackage[bookmarksnumbered, unicode, pdftex]{hyperref}

\author{Mikhail Glushenkov, \texttt{<c05mgv@cs.umu.se>}}

\title{A Cross-Platform Scalable I/O Manager for GHC}

\newcommand{\unit}[1]{\ensuremath{\, \mathrm{#1}}}

% % i Sverige har vi normalt inget indrag vid nytt stycke
\setlength{\parindent}{0pt}
% % men däremot lite mellanrum
\setlength{\parskip}{10pt}

\begin{document}
\pagestyle{empty}

\section*{Opposition Report}

\textbf{Thesis author:} Stefan Lindström\\
\textbf{Thesis topic:} Generation of test persons for the Swedish Tax Agency's
Population Registry -- An application of Context Free Grammar\\
\textbf{Main reviewer:} Mikhail Glushenkov\\
\textbf{Auxiliary reviewers:} Andreas Günzel, Jon Leijon\\
\textbf{Supervisor:} Suna Bensch\\
\textbf{Examiner:} Lars-Erik Janlert\\

All reviewers were in agreement that the thesis was overall solid, well written,
easy to follow, and reasonably complete. Discussion was initially focused on
high-level issues, after which the evaluation criteria were dealt with
point-by-point; the review was concluded with a discussion of various minor
issues.

Reviewers agreed that the main problem with the thesis was that the choice of
method (Context Free Grammars) wasn't sufficiently well motivated. It'd be nice
to see an analysis of requirements, such as who will be using the system, the
level of skill of potential users, the kinds of tasks that the users will need
to perform and so on. These requirements could then be used to evaluate the
solution that the author came up with.

Another element that the reviewers felt was missing from the thesis was an
overview of related work. Since the topic of the thesis is basically generation
of random well-formed XML documents, it would be useful and informative to see
an overview of existing work in this area. Additionally, alternative methods
could be explored, like using regular grammars instead of CFGs and using XML
Schema languages in some way.

When it comes to the ``Future Work'' section, the reviewers again felt that a
lack of clearly expressed usability and maintainability requirements undermined
the author's efforts. Since the production version of the system will most
likely need to be tweaked and modified by the users, it would be nice to see a
discussion of how these tasks could be made easier -- for example, by making the
rules more modular and/or hiding the complexity behind some kind of a high-level
domain-specific language.

One reviewer noted that the proposed solutions (RPCGs and RCGs) weren't
explained very well and that section could do well with an example; however,
another reviewer disagreed and said that the section was very clear in its
current state.

One other issue was that the status of implementation was a bit unclear from the
thesis; reviewers agreed that the implementation section was somewhat laconic
and could be expanded. For example, it'd be interesting to read more about the
kinds of problems the author had with implementing the system in Prolog and why
exactly that attempt wasn't successful.

The review then continued with going through the list of evaluation
criteria. Reviewers reiterated that the author did a good job, especially
lauding the background section when discussing the thesis's connection to
computer science. Reviewers agreed that the thesis was very readable and
well-structured, and that the result was presented well -- even though the
result is mostly negative, a negative result is still a result. It was again
noted that the choice of method wasn't very strongly motivated. One other point
that was raised was that the discussion of user privacy aspects could be
expanded a little.

Finally, minor issues like spelling, grammar, and word choice were dealt
with. One suggestion was to make the presentation of the context-free grammar in
the appendix more concise by using alternation symbols to abbreviate multiple
rules.

\end{document}
