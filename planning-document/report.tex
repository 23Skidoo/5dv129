\documentclass[10pt, oneside]{article}

\usepackage{amsmath}
\usepackage{amssymb}
\usepackage[utf8]{inputenc}
\usepackage[english]{babel}
\usepackage{enumerate}
\usepackage{titling}
\usepackage[nottoc, notlof]{tocbibind}
\usepackage[pdftex]{graphicx}
\usepackage[kerning,spacing]{microtype}
\usepackage{verbatim}
\usepackage{color}

\usepackage[bookmarksnumbered, unicode, pdftex]{hyperref}

\author{Mikhail Glushenkov, \texttt{<c05mgv@cs.umu.se>}}

\title{Planning Document}

\newcommand{\unit}[1]{\ensuremath{\, \mathrm{#1}}}

\begin{document}

\pagestyle{plain}
\pdfbookmark[1]{Framsidan}{beg}

\begin{titlingpage}
  \begin{minipage}[t]{0.45\textwidth}
  \begin{flushleft}
    \texttt{5DV129 - Degree Project: Bachelor of Science in Computing Science,
      Spring 16}
  \end{flushleft}
  \end{minipage}
  \begin{minipage}[t]{0.4\textwidth}
  \begin{flushright}
  \texttt{Umeå University}
  \end{flushright}
  \end{minipage}
  \vskip 60pt
  \begin{center}
  \LARGE\thetitle
  \par\end{center}\vskip 0.5em
  \begin{center}
  \large\theauthor
  \par\end{center}
  \begin{center}
  Date: \today
  \par\end{center}
  \vfill
  \begin{center}
    \textbf{Thesis Supervisor} \linebreak \linebreak
    Jerry Eriksson
  \end{center}
\end{titlingpage}

% % i Sverige har vi normalt inget indrag vid nytt stycke
% \setlength{\parindent}{0pt}
% % men däremot lite mellanrum
% \setlength{\parskip}{10pt}

% Start with 0.
\setcounter{section}{-1}

\section{Introduction}
\label{sec:intro}

This document contains a short description of my thesis project and a
preliminary time plan. The project is about making it possible to program
network server applications in Haskell that work on Windows and support a large
number of concurrent connections.

\section{Background}
\label{sec:background}

Haskell is a popular functional language with a powerful static type system,
focus on immutability, and a lazy (or call-by-need) evaluation strategy. The
Glasgow Haskell Compiler, GHC, is a flagship industrial-strength implementation
of Haskell. GHC is open source, compiles to fast native code on multiple
platforms, and supports a lot of extensions to the Haskell 2010
standard~\cite{bib:haskell-2010}. The extensions relevant to this project are
related to concurrency, specifically we're interested in the Concurrent Haskell
programming model~\cite{bib:marlow}.

The Concurrent Haskell programming model revolves around the concept of ``green
threads'': lightweight user-space threads that are multiplexed onto a much
smaller number of operating system (OS) threads (this number is usually set to
be equal to the number of cores). The lightweight character makes it possible to
create a large number (100K-1M) of threads; this can be used, for example, to
associate a thread with each open connection in a network server. Each
connection-bound thread then uses synchronous (blocking) I/O calls for
client-server communication. This model is nicer and more natural to program in
than an event-based one, in which all clients are served from a single thread
(one real-world example is NodeJS).

However, a foreign call that blocks a Haskell thread must not be allowed to
block the OS thread it's executing on. So the GHC's runtime system
(RTS)~\cite{bib:ghc-rts-multicore} has to do some work behind the scenes to keep
up the appearances. For foreign calls in general it uses a ``safe FFI call''
mechanism (see ~\cite{bib:ffi-conc} for more details); for the particular case
of I/O it has an I/O manager -- a dedicated thread that handles all I/O activity
and notifies blocked Haskell threads about completed I/O
operations. Interestingly, the I/O manager itself is largely written in Haskell
(unlike the rest of the RTS, which is in C).

The original implementation of the I/O manager used the venerable
\texttt{select} Unix API call, which is unsuitable for high-performance
applications. GHC version 7.0 for the first time included a new I/O
manager~\cite{bib:tibell} built upon \texttt{epoll}/\texttt{kqueue} which
improved performance and allowed I/O to scale to a much larger number of threads
than previously. In version 7.8, the implementation has been optimised even
further~\cite{bib:mio} and now scales effectively up to 48 cores.

\section{Problem Statement}
\label{sec:problem}

Sadly, the improvements described above are available only to GHC users on
Unix-like platforms due to the differences between I/O event models on Windows
(I/O completion ports) and Unix
(\texttt{epoll}/\texttt{kqueue}/\texttt{poll}). While in the Unix model the
programmer registers a number of file descriptors and waits for them to become
ready to do I/O, with I/O completion ports the programmer starts all I/O
operations asynchronously and waits for their completion. On Windows, GHC RTS
currently lacks an I/O manager and uses safe FFI calls. Besides suboptimal
performance, this also means that I/O on Windows is not interruptible: a thread
that is stuck in a blocking I/O call cannot be killed by an asynchronous
exception (e.g. timeout).

This project is essentially about fixing the GHC issue \#7353~\cite{bib:ticket}
and making the performance and scalability improvements enjoyed by users of GHC
on Unix available to Haskell programmers on Windows. The corresponding research
question is how exactly that can be achieved.

\section{Proposed Solution}
\label{sec:solution}

The proposed solution is to implement a Windows I/O manager on top of the I/O
completion ports API. There is already an existing unfinished
implementation~\cite{bib:ticket}, so the project will build upon that.

The proposed solution requires performing at least the following steps:

\begin{itemize}
\item Redesign GHC's internal APIs to support the Windows model.
\item Add support for IOCP to the I/O manager code.
\item Implement a minimal custom socket I/O library for testing.
\item Benchmark and test the new implementation.
\item Optimise the new implementation (likely by integrating with the RTS
  scheduler).
\end{itemize}

The ultimate goal of this project is to produce a patchset suitable for
inclusion in GHC, though that will likely require some further polishing after
the thesis is submitted.

\section{Time Plan}
\label{sec:time-plan}

The project will consist of two phases. During the first phase, the goal will be
to make the Windows implementation functional, but not necessarily fast. Success
criterion for this phase is working implementation of interruptible socket I/O.

During the second phase, the work will focus on making the implementation fast,
that is, practical for real-world applications. This means that it must to be no
worse than status quo on Windows. Hopefully, it will be possible to make it as
fast as the current \texttt{epoll}-based one, probably by applying ideas from
the~\cite{bib:mio} paper.

The preliminary detailed time plan looks as follows:

\begin{center}
\begin{tabular}{|l|l|}
\hline
Week 1    & Study the existing implementation, describe it in the report.\\
\hline
Week 2-3  & Port old code to a new GHC, benchmark \& test.\\
\hline
\textbf{April 25th} & Mid-project seminar: first milestone, re-evaluation.\\
\hline
Week 4-7  & Try to make the implementation practical for real applications.\\
\hline
\textbf{May 20th} & Draft submission\\
\hline
\textbf{May 23rd} & Conference\\
\hline
Week 8    & Polishing       \\
\hline
\textbf{May 30th/31st} & Disputation\\
\hline
\textbf{June 2nd} & Submission of the opposition report\\
\hline
\textbf{June 7th} & Submission of the final manuscript\\
\hline
\end{tabular}
\end{center}

\section{Risk Management}
\label{sec:risk-management}

Given that the time alotted for the project is very short, the risk management
strategy is to make the implementation as minimal as possible. Specifically, the
implementation will focus only on Windows support (the code will be
theoretically portable to other operating systems); the implementation will
focus only on socket I/O, ignoring file I/O completely; the implementation will
also use a lightweight custom library for networking instead of patching the
standard one. After the first milestone the state of the project will be
re-evaluated to see whether a change in goals or downsizing is needed.

\pagebreak

\begin{thebibliography}{9}
\label{sec:refs}

\bibitem[Haskell 2010]{bib:haskell-2010}
  \newblock
  \href{https://www.haskell.org/onlinereport/haskell2010/}{\emph{Haskell 2010
      Language Report}}\\
  \newblock Simon Marlow (ed.)\\
  \newblock Haskell.org, 2010\\

\bibitem[Tibell 2010]{bib:tibell}
  \newblock \href{http://research.google.com/pubs/pub36841.html}{\emph{Scalable I/O Event Handling for GHC}}\\
  \newblock Bryan O'Sullivan \& Johan Tibell\\
  \newblock Proceedings of the 2010 ACM SIGPLAN Haskell Symposium (Haskell'10).

\bibitem[Voellmy 2013]{bib:mio}
  \newblock \href{http://haskell.cs.yale.edu/wp-content/uploads/2013/08/hask035-voellmy.pdf}{\emph{Mio:
      A High-Performance Multicore IO
      Manager for GHC}}\\
  \newblock Andreas Voellmy et al.\\
  \newblock Proceedings of the 2013 ACM SIGPLAN symposium on Haskell
  (Haskell'13).

\bibitem[GHC Trac]{bib:ticket} \emph{GHC Trac issue \#7353}\\
  \newblock \url{https://ghc.haskell.org/trac/ghc/ticket/7353}\\
  \newblock URL accessed \today.\\

\bibitem[Marlow 2013]{bib:marlow}
  \newblock \href{http://community.haskell.org/~simonmar/pcph/}{\emph{Parallel and
      Concurrent Programming in
      Haskell}}\\
  \newblock Simon Marlow\\
  \newblock O'Reilly, 2013\\

\bibitem[Marlow 2009]{bib:ghc-rts-multicore}
  \newblock \href{http://community.haskell.org/~simonmar/papers/multicore-ghc.pdf}{\emph{Runtime
      Support for Multicore Haskell}}\\
  \newblock Simon Marlow, Simon Peyton Jones, Satnam Singh.\\
  \newblock ICFP '09: Proceeding of the 14th ACM SIGPLAN International
  Conference on Functional Programming, Edinburgh, Scotland, August 2009.\\

\bibitem[Marlow 2004]{bib:ffi-conc}
  \newblock \href{http://community.haskell.org/~simonmar/papers/conc-ffi.pdf}{\emph{Extending
      the Haskell Foreign Function Interface with Concurrency}}\\
  \newblock Simon Marlow, Simon Peyton Jones, Wolfgang Thaller\\
  \newblock Proceedings of the ACM SIGPLAN workshop on Haskell, pages 57--68, Snowbird, Utah, USA, September 2004\\

\end{thebibliography}

\end{document}
